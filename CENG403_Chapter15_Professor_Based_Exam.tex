\documentclass[12pt]{article}
\usepackage{fancyhdr}
\usepackage{amsmath,amsfonts,enumerate}
\usepackage{color,graphicx}
\usepackage{tikz}
\usepackage{pgfplots}
\usepackage{listings}
\usepackage{algorithm}
\usepackage{algorithmic}
\usetikzlibrary{arrows,positioning,shapes,calc,matrix}
\pagestyle{fancy}
%%%%%%%%%%%%%%%%%%%%%%%%%%%%%%%%%%%%%%%%%%%%%%%%%
% Course customization based on professor's lectures
%%%%%%%%%%%%%%%%%%%%%%%%%%%%%%%%%%%%%%%%%%%%%%%%%
\newcommand{\masunitnumber}{CENG 403}
\newcommand{\examdate}{January 2025}
\newcommand{\academicyear}{2024-2025}
\newcommand{\semester}{I}
\newcommand{\coursename}{Deep Learning - Large Language Models \& Vision Transformers (Professor-Based)}
\newcommand{\numberofhours}{3}
%%%%%%%%%%%%%%%%%%%%%%%%%%%%%%%%%%%%%%%%%%%%%%%%%
% CUSTOM SPACING COMMANDS FOR ANSWER SPACES
%%%%%%%%%%%%%%%%%%%%%%%%%%%%%%%%%%%%%%%%%%%%%%%%%
\newcommand{\answerspace}[1]{\vspace{#1}}
\newcommand{\questionspace}{\vspace{3cm}}        
\newcommand{\subquestionspace}{\vspace{2.5cm}}   
\newcommand{\shortanswer}{\vspace{2cm}}          
\newcommand{\mediumanswer}{\vspace{3cm}}         
\newcommand{\longanswer}{\vspace{4cm}}           
\newcommand{\journalspace}{\vspace{4.5cm}}       
\newcommand{\codespace}{\vspace{5cm}}            
%%%%%%%%%%%%%%%%%%%%%%%%%%%%%%%%%%%%%%%%%%%%%%%%%
% Header setup
%%%%%%%%%%%%%%%%%%%%%%%%%%%%%%%%%%%%%%%%%%%%%%%%%
\lhead{}
\rhead{}
\chead{{\bf MIDDLE EAST TECHNICAL UNIVERSITY}}
\lfoot{}
\rfoot{}
\cfoot{}
\begin{document}
\setlength{\headsep}{5truemm}
\setlength{\headheight}{14.5truemm}
\setlength{\voffset}{-0.45truein}
\renewcommand{\headrulewidth}{0.0pt}
\begin{center}
SEMESTER \semester\ EXAMINATION \academicyear
\end{center}
\begin{center>
{\bf \masunitnumber\ -- \coursename}
\end{center}
\vspace{20truemm}
\noindent \examdate\hspace{45truemm} TIME ALLOWED: \numberofhours\ HOURS
\vspace{19truemm}
\hrule
\vspace{19truemm}
\noindent\underline{INSTRUCTIONS TO CANDIDATES}
\vspace{8truemm}
%%%%%%%%%%%%%%%%%%%%%%%%%%%%%%%%%%%%%%%%%%%%%%%%%%%%%%
% Instructions based on lecture format
%%%%%%%%%%%%%%%%%%%%%%%%%%%%%%%%%%%%%%%%%%%%%%%%%%%%%%
\begin{enumerate}
\item This examination paper contains {\bf SIX (6)} questions and comprises 
{\bf EIGHT (8)} printed pages.
\item Answer all questions. 
The marks for each question are indicated at the beginning of each question.
\item Answer each question beginning on a {\bf FRESH} page of the answer book.
\item This {\bf IS NOT an OPEN BOOK} exam.
\item Show all mathematical derivations clearly with proper notation.
\item Draw clear diagrams with proper labels where requested.
\item Explain the intuition behind mechanisms where asked.
\end{enumerate}
%%%%%%%%%%%%%%%%%%%%%%%%%%%%%%%%%%%%%%%%%%%%%%%%%
% New page for questions
%%%%%%%%%%%%%%%%%%%%%%%%%%%%%%%%%%%%%%%%%%%%%%%%%
\newpage
\lhead{}
\rhead{\masunitnumber}
\chead{}
\lfoot{}
\cfoot{\thepage}
\rfoot{}
\setlength{\footskip}{45pt}
%%%%%%%%%%%%%%%%%%%%%%%%%%%%%%%%%%%%%%%%%%%%%%%%%%
% EXAM QUESTIONS BASED ON PROFESSOR'S LECTURES
%%%%%%%%%%%%%%%%%%%%%%%%%%%%%%%%%%%%%%%%%%%%%%%%%%

\paragraph{Question 1. Pre-training Strategies and Early Language Models}\hfill (22 marks)\\
Based on Week 15b lecture content on GPT1 and BERT pre-training.

\begin{enumerate}[(a)]
    \item The professor explained that GPT1 used "autoregressive language modeling" for pre-training. Explain this concept and describe how it differs from the supervised fine-tuning that followed. \hfill (8 marks)
    
    \mediumanswer
    
    \item According to the lecture, BERT introduced two pre-training tasks: masked language modeling and next sentence prediction. Explain both tasks and why the professor mentioned that masking creates a "discrepancy between training and testing." \hfill (8 marks)
    
    \longanswer
    
    \item The professor described how BERT uses a "CLS token" for classification tasks. Explain how this token works and why it's effective for downstream tasks. \hfill (6 marks)
    
    \mediumanswer
\end{enumerate}

\newpage
\paragraph{Question 2. Evolution of GPT Models}\hfill (25 marks)\\
Based on the professor's discussion of GPT1, GPT2, and GPT3 progression.

\begin{enumerate}[(a)]
    \item The professor noted that GPT2 showed "interesting properties emerge" when scaling up transformers. What were these emergent properties, and how did they differ from GPT1's capabilities? \hfill (8 marks)
    
    \mediumanswer
    
    \item Explain the concept of "in-context learning" as described in the lecture for GPT3. How does this differ from traditional fine-tuning approaches? \hfill (8 marks)
    
    \mediumanswer
    
    \item The professor mentioned that GPT3 came "very close to passing the Turing test." Explain what the Turing test is and analyze the significance of this achievement according to the lecture. \hfill (9 marks)
    
    \longanswer
\end{enumerate}

\newpage
\paragraph{Question 3. ChatGPT and Reinforcement Learning from Human Feedback}\hfill (20 marks)\\
Based on the professor's explanation of the three-step training process.

\begin{enumerate}[(a)]
    \item The professor described a "multi-stage training strategy" for ChatGPT. Outline the three stages and explain why each stage was necessary. \hfill (10 marks)
    
    \longanswer
    
    \item Explain how the "reward model" works as described in the lecture. How does it learn to rate responses, and why is this approach better than direct human rating? \hfill (6 marks)
    
    \mediumanswer
    
    \item The professor mentioned that ChatGPT addressed the "gap between prompt and human intention." Provide examples of this gap and explain how the training process addressed it. \hfill (4 marks)
    
    \shortanswer
\end{enumerate}

\newpage
\paragraph{Question 4. Vision Transformers}\hfill (20 marks)\\
Based on the professor's explanation of applying transformers to computer vision.

\begin{enumerate}[(a)]
    \item The professor asked "what are words in an image?" when introducing Vision Transformers. Explain the patch-based tokenization approach and why this was a key innovation. \hfill (8 marks)
    
    \mediumanswer
    
    \item According to the lecture, Vision Transformers initially "didn't perform better than CNNs" but improved with more data. Explain this finding and its implications for transformer architectures. \hfill (7 marks)
    
    \mediumanswer
    
    \item The professor described the "CLS token" approach in Vision Transformers. How does this token gather information from image patches, and how is it used for classification? \hfill (5 marks)
    
    \shortanswer
\end{enumerate}

\newpage
\paragraph{Question 5. Vision-Language Models and Multimodal Learning}\hfill (18 marks)\\
Based on the professor's discussion of combining vision and language.

\begin{enumerate}[(a)]
    \item The professor showed how to combine pre-trained LLMs with vision encoders. Explain this approach and why keeping the LLM "frozen" is beneficial. \hfill (8 marks)
    
    \mediumanswer
    
    \item Describe the vision-language action models mentioned in the lecture for robotics. How do these models process visual and textual inputs to generate robot actions? \hfill (6 marks)
    
    \mediumanswer
    
    \item The professor mentioned that "one model controlling a robot for solving different tasks" was impressive. Explain why this generalization capability is significant for robotics applications. \hfill (4 marks)
    
    \shortanswer
\end{enumerate}

\newpage
\paragraph{Question 6. Foundation Models as Agents}\hfill (15 marks)\\
Based on the professor's final discussion on using models as agents.

\begin{enumerate}[(a)]
    \item The professor introduced the concept of using foundation models as "agents" with different responsibilities. Explain how the "system prompt" enables this functionality. \hfill (6 marks)
    
    \mediumanswer
    
    \item According to the lecture, multiple LLMs can work together to solve complex tasks. Describe this collaborative approach and provide an example of how different agents might have different roles. \hfill (6 marks)
    
    \mediumanswer
    
    \item The professor mentioned using verification agents to check other agents' answers. Explain how this feedback loop approach can improve overall system performance. \hfill (3 marks)
    
    \shortanswer
\end{enumerate}

\vfill
\begin{center}{\bf END OF PAPER}\end{center>
\end{document}