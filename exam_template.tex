\documentclass[12pt]{article}
\usepackage{fancyhdr}
\usepackage{amsmath,amsfonts,enumerate}
\usepackage{color,graphicx}
\pagestyle{fancy}
%%%%%%%%%%%%%%%%%%%%%%%%%%%%%%%%%%%%%%%%%%%%%%%%%
% Do your customization here
%%%%%%%%%%%%%%%%%%%%%%%%%%%%%%%%%%%%%%%%%%%%%%%%%
\newcommand{\masunitnumber}{ACC 201}
\newcommand{\examdate}{December 2024}
\newcommand{\academicyear}{2024-2025}
\newcommand{\semester}{I}
\newcommand{\coursename}{Intermediate Accounting - Chapter 9: Accounts Receivable}
\newcommand{\numberofhours}{2}
%%%%%%%%%%%%%%%%%%%%%%%%%%%%%%%%%%%%%%%%%%%%%%%%%
% CUSTOM SPACING COMMANDS FOR ANSWER SPACES
%%%%%%%%%%%%%%%%%%%%%%%%%%%%%%%%%%%%%%%%%%%%%%%%%
% Flexible answer space command - you can adjust the size
\newcommand{\answerspace}[1]{\vspace{#1}}
% Standard spacing commands with predefined sizes
\newcommand{\questionspace}{\vspace{3cm}}        % Space between questions
\newcommand{\subquestionspace}{\vspace{2.5cm}}   % Standard space for sub-questions
\newcommand{\shortanswer}{\vspace{2cm}}          % For simple calculations
\newcommand{\mediumanswer}{\vspace{3cm}}         % For moderate complexity
\newcommand{\longanswer}{\vspace{4cm}}           % For complex problems
\newcommand{\journalspace}{\vspace{4.5cm}}       % For journal entries
%%%%%%%%%%%%%%%%%%%%%%%%%%%%%%%%%%%%%%%%%%%%%%%%%
% Don't touch anything from here till instructions
% to candidates
%%%%%%%%%%%%%%%%%%%%%%%%%%%%%%%%%%%%%%%%%%%%%%%%%
\lhead{}
\rhead{}
\chead{{\bf MIDDLE EAST TECHNICAL UNIVERSITY}}
\lfoot{}
\rfoot{}
\cfoot{}
\begin{document}
\setlength{\headsep}{5truemm}
\setlength{\headheight}{14.5truemm}
\setlength{\voffset}{-0.45truein}
\renewcommand{\headrulewidth}{0.0pt}
\begin{center}
SEMESTER \semester\ EXAMINATION \academicyear
\end{center}
\begin{center}
{\bf \masunitnumber\ -- \coursename}
\end{center}
\vspace{20truemm}
\noindent \examdate\hspace{45truemm} TIME ALLOWED: \numberofhours\ HOURS
\vspace{19truemm}
\hrule
\vspace{19truemm}
\noindent\underline{INSTRUCTIONS TO CANDIDATES}
\vspace{8truemm}
%%%%%%%%%%%%%%%%%%%%%%%%%%%%%%%%%%%%%%%%%%%%%%%%%%%%%%
% Adjust your instructions here
%%%%%%%%%%%%%%%%%%%%%%%%%%%%%%%%%%%%%%%%%%%%%%%%%%%%%%
\begin{enumerate}
\item This examination paper contains {\bf FIVE (5)} questions and comprises 
{\bf FOUR (4)} printed pages.
\item Answer all questions. 
The marks for each question are indicated at the beginning of each question.
\item Answer each question beginning on a {\bf FRESH} page of the answer book.
\item This {\bf IS NOT an OPEN BOOK} exam.
\item Candidates may use calculators. However, they should write down systematically the steps in the workings.
\item Show all journal entries with proper account titles and amounts.
\item Use proper accounting format: Account Title followed by dollar amount, with debits listed first.
\item Round all calculations to the nearest dollar unless otherwise specified.
\end{enumerate}
%%%%%%%%%%%%%%%%%%%%%%%%%%%%%%%%%%%%%%%%%%%%%%%%%
% leave this as it is
%%%%%%%%%%%%%%%%%%%%%%%%%%%%%%%%%%%%%%%%%%%%%%%%%
\newpage
\lhead{}
\rhead{\masunitnumber}
\chead{}
\lfoot{}
\cfoot{\thepage}
\rfoot{}
\setlength{\footskip}{45pt}
%%%%%%%%%%%%%%%%%%%%%%%%%%%%%%%%%%%%%%%%%%%%%%%%%%
% EXAM QUESTIONS WITH ANSWER SPACES
%%%%%%%%%%%%%%%%%%%%%%%%%%%%%%%%%%%%%%%%%%%%%%%%%%
\paragraph{Question 1. Basic Accounts Receivable Transactions}\hfill (20 marks)\\
RAS Co. had the following transactions during July:
\begin{itemize}
\item July 1: Sold merchandise on account to Waegelein Inc. for \$17,200, terms 2/10, n/30.
\item July 8: Waegelein Inc. returned merchandise worth \$3,800 to RAS Co.
\item July 11: Waegelein Inc. paid for the merchandise.
\end{itemize}

\begin{enumerate}[(a)]
    \item Record the journal entry for the sale on July 1. \hfill (7 marks)
    
    \journalspace
    
    \item Record the journal entry for the merchandise return on July 8. \hfill (6 marks)
    
    \journalspace
    
    \item Record the journal entry for the payment on July 11 (assume payment is within discount period). \hfill (7 marks)
    
    \journalspace
\end{enumerate}

\newpage
\paragraph{Question 2. Allowance Method and Financial Statement Presentation}\hfill (25 marks)\\
During its first year of operations, Gavin Company had credit sales of \$3,000,000. At year-end, \$600,000 remained uncollected. The credit manager estimates that \$31,000 of these receivables will become uncollectible. In addition to the receivables, the company has cash of \$90,000, inventory of \$130,000, and prepaid insurance of \$7,500.

\begin{enumerate}[(a)]
    \item Prepare the journal entry to record the estimated uncollectibles. \hfill (8 marks)
    
    \journalspace
    
    \item Prepare the current assets section of the balance sheet for Gavin Company. Assume that in addition to the receivables, the company has cash of \$90,000, inventory of \$130,000, and prepaid insurance of \$7,500. \hfill (17 marks)
    
    \longanswer
\end{enumerate}

\questionspace
\paragraph{Question 3. Write-off and Cash Realizable Value}\hfill (20 marks)\\
At the end of 2017, Carpenter Co. has accounts receivable of \$700,000 and an allowance for doubtful accounts of \$54,000. On January 24, 2018, the company learns that its receivable from Megan Gray is not collectible, and management authorizes a write-off of \$6,200.

\begin{enumerate}[(a)]
    \item Prepare the journal entry to record the write-off. \hfill (8 marks)
    
    \journalspace
    
    \item What is the cash realizable value of the accounts receivable (1) before the write-off and (2) after the write-off? \hfill (12 marks)
    
    \mediumanswer
\end{enumerate}

\newpage
\paragraph{Question 4. Multiple Transaction Analysis}\hfill (25 marks)\\
Two independent situations are presented below:

\textbf{Situation A:} On January 6, Brumbaugh Co. sells merchandise on account to Pryor Inc. for \$7,000, terms 2/10, n/30. On January 16, Pryor Inc. pays the amount due. Prepare the entries on Brumbaugh's books to record the sale and related collection.

\textbf{Situation B:} On January 10, Andrew Farley uses his Paltrow Co. credit card to purchase merchandise from Paltrow Co. for \$9,000. On February 10, Farley is billed for the amount due of \$9,000. On February 12, Farley pays \$5,000 on the balance due. On March 10, Farley is billed for the amount due, including interest at 1\% per month on the unpaid balance as of February 12. Prepare the entries on Paltrow Co.'s books related to the transactions that occurred on January 10, February 12, and March 10.

\begin{enumerate}[(a)]
    \item Prepare journal entries for Situation A. \hfill (10 marks)
    
    \journalspace
    
    \item Prepare journal entries for Situation B. \hfill (15 marks)
    
    \answerspace{6cm}
\end{enumerate}

\questionspace
\paragraph{Question 5. Allowance for Doubtful Accounts - Different Methods}\hfill (30 marks)\\
The ledger of Costello Company at the end of the current year shows Accounts Receivable \$110,000, Sales Revenue \$840,000, and Sales Returns and Allowances \$20,000.

\textbf{Instructions:}
\begin{enumerate}[(a)]
    \item If Costello uses the direct write-off method to account for uncollectible accounts, journalize the adjusting entry at December 31, assuming Costello determines that L. Dole's \$1,400 balance is uncollectible. \hfill (8 marks)
    
    \journalspace
    
    \item If Allowance for Doubtful Accounts has a credit balance of \$2,100 in the trial balance, journalize the adjusting entry at December 31, assuming bad debts are expected to be (1) 1\% of net sales, and (2) 10\% of accounts receivable. \hfill (14 marks)
    
    \answerspace{5cm}
    
    \item If Allowance for Doubtful Accounts has a debit balance of \$200 in the trial balance, journalize the adjusting entry at December 31, assuming bad debts are expected to be (1) 0.75\% of net sales and (2) 6\% of accounts receivable. \hfill (8 marks)
    
    \journalspace
\end{enumerate}

\vfill
\begin{center}{\bf END OF PAPER}\end{center}
\end{document}